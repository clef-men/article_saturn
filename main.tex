\documentclass[a4paper, 11pt]{article}

\usepackage[top=0.7in, bottom=0.7in]{geometry}

%------------------------------------------------------------

\usepackage[utf8]{inputenc}
\usepackage[T1]{fontenc}
\usepackage[english]{babel}

%------------------------------------------------------------

\usepackage{multirow}

\usepackage[dvipsnames]{xcolor}

\usepackage{hyperref}
% see https://en.wikibooks.org/wiki/LaTeX/Colors
\hypersetup{breaklinks=true,colorlinks=true,citecolor=OliveGreen,urlcolor=Plum,linkcolor=Purple}

\usepackage{colortbl}
\usepackage{xcolor}
%------------------------------------------------------------

\usepackage[style=numeric, sorting=ynt, doi=false, maxnames=10]{biblatex}
\addbibresource{main.bib}

%------------------------------------------------------------

\usepackage{macros}

%------------------------------------------------------------

\title{\Saturn: a library of verified concurrent data structures for \OCaml~5}
\date{\today}
\author{
  Clément Allain (INRIA) \\
  Vesa Karvonen (Tarides) \\
  Carine Morel (Tarides)
}

%------------------------------------------------------------
%------------------------------------------------------------

\begin{document}

\maketitle

\section{Abstract}

We present \Saturn, a new \OCaml~5 library available on \opam.
\Saturn offers a collection of efficient concurrent data structures: stack, queue, skiplist, hash table, work-stealing deque, etc.
It is well tested, benchmarked and in part formally verified.

\section{Motivation}

Sharing data between multiple threads or cores is a well-known problem.
A naive approach is to take a sequential data structure and protect it with a lock.
However, this is often not the best solution.
First, if performance is a concern, this approach is likely to be inefficient as locks introduce significant contention.
Second, it may not be a sound solution as it can lead to liveness issues such as deadlock, starvation and priority inversion.

In contrast, \emph{lock-free} implementations relying on fine-grained synchronisation rather than locks are typically faster and guaranty system-wide progress.
Yet, they are also more complex and come with their own set of bugs: ABA problem (largely mitigated in garbage-collected languages), data races, unexpected behaviors due to non-linearizability.

In this context, \Saturn provides a collection of standard lock-free data structures saving \OCaml~5 programmers the trouble of designing their own.

\section{Library design}

\Saturn aims at covering a wide range of use cases, from simple stacks and queues to more complex data structures like skiplists and hash tables.
More precisely, it currently features:
(A) numerous queues: a queue based on the well-known Michael-Scott queue~\cite{michael1996simple}, a single-producer single-consumer queue, a multiple-producer single-consumer queue and a bounded queue;
(B) a stack based on the Treiber stack~\cite{treiber1986systems};
(C) a work-stealing deque;
(D) a bag;
(E) a hash table;
(F) a skiplist.

% How we choose the data structures and algorithms
Most implementations are based on well-known algorithms.
They have been adapted to work with and take advantage of the \OCaml~5 memory model.
For instance, we had to rework the Michael-Scott queue to avoid memory leaks.

% Optimizations
Regarding performance, we are working on providing benchmarks for each \Saturn's data structure (see section~\ref{sec:benchmarks}), and significant effort has been dedicated to micro-optimization. In particular, we worked on
(A) preventing false sharing\footnote[1]{False sharing occurs when different domains access different data items contained in the same cache line, forcing unnecessary synchronization. To prevent this, these data must be padded to ensure they are not in the same cache line.},
(B) adding fenceless atomic reads when possible, which improves performance on ARM processors, and
(C) avoiding the extra indirection in arrays of atomics to reduce memory consumption.
The feedback we produced while optimizing \Saturn has highlighted some missing features in \OCaml~5 and led to improvements in upstream \OCaml (\href{https://github.com/ocaml/ocaml/pull/12212}{padded atomics}, \href{https://github.com/ocaml/ocaml/pull/12715}{CSE bug fixed}).

% Standard and optimized versions
To explore some of these optimizations, we use unsafe features of the language (e.g., \texttt{Obj.magic}). Although we design our code to be memory-safe under regular use (e.g. only one domain can push at any given time in a single-consumer single-producer queue), we cannot offer the same level of guarantee as with the standard implementations. Consequently, some of \Saturn's data structures have two versions: (1) a version that does not use any unsafe features of \OCaml and (2) an optimized version. While most users should find the regular version efficient enough for their needs, adventurous users may prefer the optimized version, provided they encapsulate it correctly and verify their code somehow.

% \section{Example of use}
% It feels it would be nice to have a simple example here to show that it is easy to use the library.

% Note Possible question :  
% - When it should not be used ?  
%   - composability -> see Kcas
%   - very specific needs and performance
% - Is it ready ?

% 
\section{Benchmarks}\label{sec:benchmarks}
% As we are still in the experimental phase, we only provide some rough preliminary numbers to give an idea of the library performance. The following table shows the throughput of three implementations of a single-consumer single-producer queue in million messages per second. The \href{https://github.com/ocaml-multicore/saturn/blob/main/src_lockfree/spsc_queue.ml}{first implementation} is provided by \Saturn under the name \texttt{Single\_cons\_single\_prod\_queue}. The second one is an optimized version that uses unsafe features of \OCaml. The \href{https://github.com/ocaml-multicore/saturn/pull/133}{last one} is written in \Cpp and provides a good upper limit. The tests were run on an Intel i7-1270P and an Arm Cortex M3.

% \hspace{0.5cm}

% \begin{tabular}{r r r}
%   \textbf{Throughput (million message/s)} & \textbf{Intel} & \textbf{Arm} \\
%   \hline
%   SPSC queue                              & 60 M/s         & 100 M/s      \\
%   Optimized SPSC queue                    & 100 M/s        & 250 M/s      \\
%   \Cpp SPSC queue                         & 400 M/s        & 500 M/s      \\
% \end{tabular}

As we are still in the experimental phase, we provide rough preliminary numbers to give an idea of the library performance. The following tables show the throughput of various queues and stacks implementations. The queue implementations benchmarked are:
(1) the \Stdlib queue (with one domain only),
(2) the \Stdlib queue protected with a mutex,
(3) the lock-free Michael-Scott queue from \Saturn,
(4) a Michael-Scott two-stack-based queue (currently in this \href{https://github.com/ocaml-multicore/saturn/pull/112}{PR} in \Saturn).
The stack implementations benchmarked are
(1) the \Stdlib stack (with one domain only),
(2) the \Stdlib stack protected with a mutex,
(3) a concurrent stack implemented with an atomic list,
(4) a lock-free Treiber stack from \Saturn.
The tests were run on an Intel i7-1270P (4P+8E cores) and an Apple M3 Max (6P+6P+4E cores) using OCaml 5.2.0 (see this \href{https://github.com/lyrm/saturn_benchmarks}{repository} if you want to run your own benchmarks).


\begin{table}[htbp]
  \centering
  \begin{tabular}{|l|l|l|}
    \hline
    \textbf{Queue Implementation} & \textbf{Intel i7-1270P} & \textbf{Apple M3} \\ \hline
    Stdlib queue                  & \best{61 M/s}           & \best{64 M/s}     \\ \hline
    Stdlib queue with mutex       & 24 M/s                  & \worst{19 M/s}    \\ \hline
    Saturn Michael-Scott queue    & \worst{22 M/s}          & 32 M/s            \\ \hline
    Two\_stack queue              & 37 M/s                  & 56 M/s            \\ \hline
  \end{tabular}
  \caption{Message over time (million messages per second) for several queue implementations on a single domain}
  \label{tab:queue-benchmarks-one-domain}
\end{table}

There are many insights to be drawn from these results, but we will highlight a few key points. Firstly, for sequential programs, the \Stdlib queue and stack are the fastest implementations as the concurrent implementations add significant overhead. However, the \Saturn implementations largely outperform the \Stdlib ones protected with a single lock, even under low contention. Finally, the concurrent stack implemented with an atomic list performs comparably to the Treiber stack\footnote[2]{The Treiber stack is essentially an atomic list.}. 
There is still a benefit to using \Saturn data structures in this case: even this basic implementation is optimized through
(a) the use of \texttt{make\_contended} to prevent false sharing, and
(b) a backoff mechanism to reduce contention.
Without these seemingly small optimizations, the atomic list implementation has a throughput of around 10 M/s regardless of contention, which is significantly lower than the Treiber stack's performance. 


\begin{table}[htbp]
  \centering
  \begin{tabular}{|l|l|l|}
    \hline
    \textbf{Stack Implementation} & \textbf{Intel i7-1270P} & \textbf{Apple M3} \\ \hline
    Stdlib stack                  & \best{66 M/s}           & \best{72 M/s}     \\ \hline
    Stdlib stack with mutex       & \worst{24 M/s}          & \worst{24 M/s}    \\ \hline
    Atomic list                   & 52 M/s                  & 66 M/s            \\ \hline
    Saturn Treiber stack          & 47 M/s                  & 67 M/s            \\ \hline
  \end{tabular}
  \caption{Message over time (million messages per second) for several stack implementations on a single domain}
  \label{tab:stack-benchmarks-one-domain}
\end{table}

\begin{table}[htbp]
  \centering
  \begin{tabular}{|l|l|l|l|}
    \hline
    \textbf{Configuration}              & \textbf{Queue Implementation} & \textbf{Intel i7-1270P} & \textbf{Apple M3} \\ \hline
    \multirow{3}{*}{1 adder, 1 taker}   & Stdlib queue with mutex       & \worst{6.1 M/s}         & \worst{14 M/s}    \\ \cline{2-4}
                                        & Saturn Michael-Scott queue    & 19 M/s                  & 45 M/s            \\ \cline{2-4}
                                        & Two\_stack queue              & \best{40 M/s}           & \best{110 M/s}    \\ \hline
    \multirow{3}{*}{1 adder, 2 takers}  & Stdlib queue with mutex       & \worst{3.1 M/s}         & \worst{3.2 M/s}   \\ \cline{2-4}
                                        & Saturn Michael-Scott queue    & 18 M/s                  & 16 M/s            \\ \cline{2-4}
                                        & Two\_stack queue              & \best{36 M/s}           & \best{102 M/s}    \\ \hline
    \multirow{3}{*}{2 adders, 1 taker}  & Stdlib queue with mutex       & \worst{5.8 M/s}         & \worst{5.8 M/s}   \\ \cline{2-4}
                                        & Saturn Michael-Scott queue    & 9.9 M/s                 & 24 M/s            \\ \cline{2-4}
                                        & Two\_stack queue              & \best{17 M/s}           & \best{89 M/s}     \\ \hline
    \multirow{3}{*}{2 adders, 2 takers} & Stdlib queue with mutex       & \worst{3.6 M/s}         & \worst{6.0 M/s}   \\ \cline{2-4}
                                        & Saturn Michael-Scott queue    & 8.2 M/s                 & 29 M/s            \\ \cline{2-4}
                                        & Two\_stack queue              & \best{17 M/s}           & \best{97 M/s}     \\ \hline
  \end{tabular}
  \caption{Message over time (million messages per second) for several stack implementations on a multiple domain running in parallel}
  \label{tab:queue-benchmarks-parallel}
\end{table}

\begin{table}[htbp]
  \centering
  \begin{tabular}{|l|l|l|l|}
    \hline
    \textbf{Configuration}              & \textbf{Stack Implementation} & \textbf{Intel i7-1270P} & \textbf{Apple M3} \\ \hline
    \multirow{4}{*}{1 adder, 1 taker}   & Stdlib stack with mutex       & \worst{2.7 M/s}         & \worst{18 M/s}        \\ \cline{2-4}
                                        & Atomic list                   & 66 M/s                  & \best{140 M/s}        \\ \cline{2-4}
                                        & Saturn Treiber stack          & \best{70 M/s}           & 128 M/s               \\ \hline
    \multirow{4}{*}{1 adder, 2 takers}  & Stdlib stack with mutex       & \worst{3.1 M/s}         & \worst{4.0 M/s}       \\ \cline{2-4}
                                        & Atomic list                   & \best{49 M/s}           & 113 M/s               \\ \cline{2-4}
                                        & Saturn Treiber stack          & 46 M/s                  & \best{104 M/s}        \\ \hline
    \multirow{4}{*}{2 adders, 1 taker}  & Stdlib stack with mutex       & \worst{6.5 M/s}         & \worst{7.7 M/s}       \\ \cline{2-4}
                                        & Atomic list                   & 52 M/s                  & \best{120 M/s}        \\ \cline{2-4}
                                        & Saturn Treiber stack          & \best{60 M/s}           & 114 M/s               \\ \hline
    \multirow{4}{*}{2 adders, 2 takers} & Stdlib stack with mutex       & \worst{3.6 M/s}         & \worst{7.7 M/s}       \\ \cline{2-4}
                                        & Atomic list                   & 41 M/s                  & \best{107 M/s}        \\ \cline{2-4}
                                        & Saturn Treiber stack          & \best{43 M/s}           & 99 M/s                \\ \hline
  \end{tabular}
  \caption{Messages over time (million messages per second) for several stack implementations running in parallel}
  \label{tab:stack-benchmarks-parallel}
\end{table}

\section{Tests}

In multicore programming, it is essential to test not only the safety of the data structures but also to verify linearizability and lock-freedom when expected.
To achieve this, \Saturn has been thoroughly tested using two primary tools: \DSCheck and \STM.

\STM is used not only for unit testing but also for linearizability.
It automatically generates random full programs using the provided API---in the case of \Saturn, a data structure.
These programs are executed in parallel with two domains and all results are checked against the postconditions of each function, providing unit testing.
Simultaneously, \STM verifies linearizability by ensuring that all intermediate states can be explained by a sequential execution of the calls. The STM \href{https://github.com/ocaml-multicore/saturn/blob/main/test/treiber_stack/stm_treiber_stack.ml}{test for the Treiber stack} are a good example of how simple this is to write.

\DSCheck is a model checker based on the DPOR\footnote[3]{DPOR stands for Dynamic Partial-Order Reduction} algorithm~\cite{dpor05}.
It is designed to compute all possible interleavings of a given program and verify that each one returns the expected result.
This is particularly useful for catching elusive bugs that occur only in specific, rare interleavings.
Additionally, \DSCheck can be used to verify that a program is lock-free, as it will fail to terminate if any form of blocking is present. This is a bit more cumbersome to use than \STM (see the \href{https://github.com/ocaml-multicore/saturn/blob/main/test/treiber_stack/treiber_stack_dscheck.ml}{DSCheck tests for the Treiber stack}) but it is still a powerful tool. \DSCheck implementation has been optimized\footnote[4]{See the PRs about \href{https://github.com/ocaml-multicore/dscheck/pull/18}{source sets} and \href{https://github.com/ocaml-multicore/dscheck/pull/22}{granular dependency relation}.} to make the tests quick enough to be used even on the more complex data structures of \Saturn (see \href{https://github.com/ocaml-multicore/saturn/blob/main/test/skiplist/stm_skiplist.ml}{the skiplist DSCheck tests}).

\section{Formal verification}

Lock-free algorithms are notoriously difficult to get right.
To provide stronger guarantees, we have verified part of \Saturn's data structures and aim at covering the entire library.
Moreover, one important benefit is that we get formal specifications for verified data structures.

This verification effort has been conducted using \Iris~\cite{DBLP:journals/jfp/JungKJBBD18}, a state-of-the-art mechanized \emph{concurrent separation logic}.
In particular, all proofs are formalized in \Coq.

A common criterion to specify concurrent operations on shared data structures is \emph{linearizability}.
The equivalent \Iris notion is \emph{logical atomicity}: there exists a point in time when a concurrent operation atomically takes effect (the linearization point).
This statement takes the form of an \emph{atomic specification}:

\[
  \aspec{
    \mathrm{queue \mathhyphen inv}\ t
  }{
    \mathit{vs}
  }{
    \mathrm{queue \mathhyphen model}\ t\  \mathit{vs}
  }{
    \texttt{queue\_push}\ t\ v
  }{
    \mathrm{queue \mathhyphen model}\ t\  (\mathit{vs} \mdoubleplus [v])
  }{
    \texttt{()}
  }{
    \mathrm{True}
  }
\]

In this example, we specify the \texttt{queue\_push} operation from an implementation of a concurrent queue.
Similarly to \href{https://en.wikipedia.org/wiki/Hoare_logic}{Hoare triples}, the two assertions inside curly brackets express the precondition and postcondition.
Here, the $\mathrm{queue \mathhyphen inv}\ t$ precondition represents the queue invariant.
As it is persistent, we do not need to give it back in the postcondition.
The other two assertions inside angle brackets express the \emph{atomic precondition} and \emph{atomic postcondition}.
Basically, they specify the linearization point of the operation: during the execution of \texttt{queue\_push}, the logical model of the queue is atomically updated from $\mathit{vs}$ to $\mathit{vs} \mdoubleplus [v]$, in other words $v$ is atomically pushed at the back of the queue.

As a final note, we emphasize that our verification assumes a sequentially consistent memory model.
Nevertheless, \OCaml~5's relaxed memory model has been formalized~\cite{DBLP:journals/pacmpl/MevelJP20} in \Iris.
It should be possible and is future work to adapt our specifications and proofs to support it.

% \section{Futur work}
% By providing extensively tested, formally verified, and benchmarked concurrent implementations, \Saturn aims to help \OCaml~5 users avoid the pitfalls and intricacies of implementing their own concurrent data structures. The library is available on \opam and is under active development. We plan to add new data structures (such as a lock-free priority queue, a bounded stack, and a lock-free set, among others) and continue improving the existing ones. Additionally, we are working on formally verifying even more data structures using \Iris. 

% \section{Talk proposal}

% In our talk, we will introduce \Saturn, including the main data structures and design guidelines.
% We will also discuss ongoing work on verifying the library using the \Iris concurrent separation logic.

\printbibliography

\end{document}
